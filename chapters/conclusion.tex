\chapter{Conclusión}

El trabajo realizado cumplió con el objetivo principal, es decir la creación de tres programas de cómputo científico que calculan entalpías de formación de compuestos orgánicos a \textit{T} = 298.15 K con correcciones opcionales a la energía interna utilizando aproximaciones como Nicolaides \textit{et al.}, rotor rígido y oscilador armónico, empleando archivos de salida de Gaussian09. Además, se activaron otros métodos con un alto nivel de teoría y conjuntos de bases pequeñas, los cuales fueron: G3, G3MP2, CBS-APNO, CBS-QB3 y G4. De igual forma, se añadieron valores experimentales reportados por la comunidad científica: Tajti \textit{et al.}, NIST y Argonne.\\

También, se lograron otros objetivos específicos:

\begin{itemize}
\item Los programas fueron diseñados para utilizarse a través de la línea de comandos en un sistema operativo de GNU/Linux, dando como resultado, una alta eficiencia en el flujo de trabajo.

\item Se implementó una programación orientada a objetos que fragmentó el código en partes independientes, permitiendo así, reciclar el código para proyectos futuros.
\end{itemize}





